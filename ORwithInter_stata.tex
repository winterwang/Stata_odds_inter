% Options for packages loaded elsewhere
\PassOptionsToPackage{unicode}{hyperref}
\PassOptionsToPackage{hyphens}{url}
%
\documentclass[
]{article}
\usepackage{lmodern}
\usepackage{amssymb,amsmath}
\usepackage{ifxetex,ifluatex}
\ifnum 0\ifxetex 1\fi\ifluatex 1\fi=0 % if pdftex
  \usepackage[T1]{fontenc}
  \usepackage[utf8]{inputenc}
  \usepackage{textcomp} % provide euro and other symbols
\else % if luatex or xetex
  \usepackage{unicode-math}
  \defaultfontfeatures{Scale=MatchLowercase}
  \defaultfontfeatures[\rmfamily]{Ligatures=TeX,Scale=1}
\fi
% Use upquote if available, for straight quotes in verbatim environments
\IfFileExists{upquote.sty}{\usepackage{upquote}}{}
\IfFileExists{microtype.sty}{% use microtype if available
  \usepackage[]{microtype}
  \UseMicrotypeSet[protrusion]{basicmath} % disable protrusion for tt fonts
}{}
\makeatletter
\@ifundefined{KOMAClassName}{% if non-KOMA class
  \IfFileExists{parskip.sty}{%
    \usepackage{parskip}
  }{% else
    \setlength{\parindent}{0pt}
    \setlength{\parskip}{6pt plus 2pt minus 1pt}}
}{% if KOMA class
  \KOMAoptions{parskip=half}}
\makeatother
\usepackage{xcolor}
\IfFileExists{xurl.sty}{\usepackage{xurl}}{} % add URL line breaks if available
\IfFileExists{bookmark.sty}{\usepackage{bookmark}}{\usepackage{hyperref}}
\hypersetup{
  pdftitle={Perform logistic regression model with interaction between binary variables in Stata},
  pdfauthor={Chaochen Wang \textbar{} 王 超辰 Hiroshi Yatsuya \textbar{} 八谷 寛},
  hidelinks,
  pdfcreator={LaTeX via pandoc}}
\urlstyle{same} % disable monospaced font for URLs
\usepackage[margin=1in]{geometry}
\usepackage{graphicx}
\makeatletter
\def\maxwidth{\ifdim\Gin@nat@width>\linewidth\linewidth\else\Gin@nat@width\fi}
\def\maxheight{\ifdim\Gin@nat@height>\textheight\textheight\else\Gin@nat@height\fi}
\makeatother
% Scale images if necessary, so that they will not overflow the page
% margins by default, and it is still possible to overwrite the defaults
% using explicit options in \includegraphics[width, height, ...]{}
\setkeys{Gin}{width=\maxwidth,height=\maxheight,keepaspectratio}
% Set default figure placement to htbp
\makeatletter
\def\fps@figure{htbp}
\makeatother
\setlength{\emergencystretch}{3em} % prevent overfull lines
\providecommand{\tightlist}{%
  \setlength{\itemsep}{0pt}\setlength{\parskip}{0pt}}
\setcounter{secnumdepth}{5}
\usepackage{bookmark}
\usepackage{xltxtra}
\usepackage{zxjatype}
\usepackage[ipaex]{zxjafont}

\title{Perform logistic regression model with interaction between binary
variables in Stata}
\author{Chaochen Wang \textbar{} 王 超辰 \and Hiroshi Yatsuya
\textbar{} 八谷 寛}
\date{2020-10-22 14:46:38 JST created, 2020-10-22 16:13:02 updated}

\begin{document}
\maketitle

This demonstration can be found with more detailed explanation about how
to visually show the interaction effect among categorical variables from
the UCLA website:

\url{https://stats.idre.ucla.edu/stata/faq/how-can-i-understand-a-categorical-by-categorical-interaction-in-logistic-regression-stata-12/}

You can use the complete command to load the data into your Stata
environment:

\begin{verbatim}
 use https://stats.idre.ucla.edu/stat/data/logit2-2, clear
\end{verbatim}

The example dataset called logit2-2 includes two binary variables,
\texttt{f} and \texttt{h}, and a continuous variable as covariate
\texttt{cv1}. We build a model include \texttt{f} by \texttt{h}
interaction, with the covariate \texttt{cv1}.

\begin{verbatim}
## 
## . use https://stats.idre.ucla.edu/stat/data/logit(highschool and beyond (200 cases))
## 
## . logistic y f##h cv1
## 
## Logistic regression                             Number of obs     =        200
##                                                 LR chi2(4)        =     106.10
##                                                 Prob > chi2       =     0.0000
## Log likelihood =  -78.74193                     Pseudo R2         =     0.4025
## 
## ------------------------------------------------------------------------------
##            y | Odds Ratio   Std. Err.      z    P>|z|     [95% Conf. Interval]
## -------------+----------------------------------------------------------------
##          1.f |   20.00771   15.04885     3.98   0.000      4.58104    87.38374
##          1.h |   10.92345   7.218757     3.62   0.000     2.991185     39.8911
##              |
##          f#h |
##         1 1  |   .1290242   .1136444    -2.32   0.020      .022958    .7251177
##              |
##          cv1 |   1.217106   .0399841     5.98   0.000     1.141208    1.298052
##        _cons |   7.06e-06   .0000134    -6.26   0.000     1.72e-07    .0002902
## ------------------------------------------------------------------------------
## Note: _cons estimates baseline odds.
## 
## .
\end{verbatim}

As you can see all of the variables in the above model including the
interaction term are statistically significant. Which means the
coefficients in what we fitted in the above model were all statistically
significant. The model can be written as below:

\[
\text{logit}(Pr(y = 1)) = \alpha + \beta_1 f_{i} + \beta_2 h_{i} + \beta_3 f_i\times h_i + \beta_4 \text{cv1}
\]

We store the above results in object called \texttt{inter}. And build
another model without the interaction term (as \texttt{main}) and use
\texttt{lrtest} commend to test the significance of the interaction.
Note that the \texttt{quietly} is to suppress the output of the
\texttt{inter} model to save space. For completeness, we will also use a
Wald test (\texttt{test} command). But we know that a Wald test is an
approximation to the likelihood ratio test (\texttt{lrtest}), the LRtest
is preferred.

\begin{verbatim}
## 
## . use https://stats.idre.ucla.edu/stat/data/logit(highschool and beyond (200 cases))
## 
## . logistic y i.f i.h cv1
## 
## Logistic regression                             Number of obs     =        200
##                                                 LR chi2(3)        =     100.26
##                                                 Prob > chi2       =     0.0000
## Log likelihood =   -81.6618                     Pseudo R2         =     0.3804
## 
## ------------------------------------------------------------------------------
##            y | Odds Ratio   Std. Err.      z    P>|z|     [95% Conf. Interval]
## -------------+----------------------------------------------------------------
##          1.f |   5.215943    2.20634     3.90   0.000      2.27654    11.95062
##          1.h |   3.513298   1.408747     3.13   0.002     1.601046    7.709499
##          cv1 |   1.197961   .0364223     5.94   0.000      1.12866    1.271518
##        _cons |   .0000347   .0000563    -6.33   0.000     1.44e-06    .0008345
## ------------------------------------------------------------------------------
## Note: _cons estimates baseline odds.
## 
## . estimates store main
## 
## . quietly logistic y i.f##i.h cv1
## 
## . estimates store inter
## 
## . lrtest main inter 
## 
## Likelihood-ratio test                                 LR chi2(1)  =      5.84
## (Assumption: main nested in inter)                    Prob > chi2 =    0.0157
## 
## . 
## . test 1.f#1.h 
## 
##  ( 1)  [y]1.f#1.h = 0
## 
##            chi2(  1) =    5.41
##          Prob > chi2 =    0.0201
## 
## .
\end{verbatim}

\newpage

Since the interaction effect is significant, we will use the
\texttt{inter} model to obtain our odds ratios with confidence intervals
through \texttt{lincom} (linear combination of parameters) command.

\begin{verbatim}
## . lincom 1.f, eform
## 
##  ( 1)  [y]1.f = 0
## 
## ------------------------------------------------------------------------------
##            y |     exp(b)   Std. Err.      z    P>|z|     [95% Conf. Interval]
## -------------+----------------------------------------------------------------
##          (1) |   20.00771   15.04885     3.98   0.000      4.58104    87.38374
## ------------------------------------------------------------------------------
\end{verbatim}

\begin{verbatim}
## . lincom 1.f 1.f#1.h, eform
##  ( 1)  [y]1.f + [y]1.f#1.h = 0
## 
## ------------------------------------------------------------------------------
##            y |     exp(b)   Std. Err.      z    P>|z|     [95% Conf. Interval]
## -------------+----------------------------------------------------------------
##          (1) |   2.581479   1.319015     1.86   0.063     .9482971    7.027367
## ------------------------------------------------------------------------------
\end{verbatim}

Therefore,

\begin{itemize}
\tightlist
\item
  the OR for \texttt{f} = 1 vs.~\texttt{f} = 0 when h = 0 and
  controlling for \texttt{cv1} is 20.1 (95\% CI: 4.58, 87.4);
\item
  the OR for \texttt{f} = 1 vs.~\texttt{f} = 0 when h = 1 and
  controlling for \texttt{cv1} is 2.58 (95\% CI: 0.95, 7.03).
\end{itemize}

\begin{verbatim}
## . lincom 1.h, eform
## 
##  ( 1)  [y]1.h = 0
## 
## ------------------------------------------------------------------------------
##            y |     exp(b)   Std. Err.      z    P>|z|     [95% Conf. Interval]
## -------------+----------------------------------------------------------------
##          (1) |   10.92345   7.218757     3.62   0.000     2.991185     39.8911
## ------------------------------------------------------------------------------


##  . lincom 1.h + 1.f#1.h, eform
##
##  ( 1)  [y]1.h + [y]1.f#1.h = 0
##
## ------------------------------------------------------------------------------
##            y |     exp(b)   Std. Err.      z    P>|z|     [95% Conf. Interval]
## -------------+----------------------------------------------------------------
##          (1) |   1.409389   .7762522     0.62   0.533     .4788645    4.148098
## ------------------------------------------------------------------------------
\end{verbatim}

Therefore,

\begin{itemize}
\tightlist
\item
  the OR for \texttt{h} = 1 vs.~\texttt{h} = 0 when f = 0 and
  controlling for \texttt{cv1} is 10.9 (95\% CI: 2.99, 39.9);
\item
  the OR for \texttt{h} = 1 vs.~\texttt{h} = 0 when f = 1 and
  controlling for \texttt{cv1} is 1.41 (95\% CI: 0.48, 4.15).
\end{itemize}

\end{document}
